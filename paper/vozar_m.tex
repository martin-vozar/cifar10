\documentclass[12pt,a4paper]{article}

\usepackage{times}
\usepackage[T1]{fontenc}
\usepackage{graphicx}
\usepackage{subfig}
\usepackage{caption}
\usepackage{epsfig}
\usepackage{epstopdf}
\usepackage{fancyhdr}
\usepackage[dvipsnames]{color}
\usepackage{hyperref}
\usepackage{amsmath}
\usepackage{multicol}
% \usepackage[slovak]{babel}
\usepackage{float}

\usepackage[backend=bibtex, style=numeric, citestyle=authoryear, sorting=none]{biblatex}
% \addbibresource{sources.bib}
\renewcommand*{\nameyeardelim}{\addcomma\addspace}

\textwidth = 170.0mm
\textheight = 246.0mm

\topskip=5mm
\topmargin=0mm
\evensidemargin=-5.4mm
\oddsidemargin=-5.4mm
\headheight=14pt
\headsep=5mm

\parindent=0mm
\parskip  =2mm

\pagestyle{fancy}

\fancyfoot[C]{}

\voffset=-1.25cm
% \pagenumbering{arabic}
% \setcounter{page}{1}
\pagestyle{plain}
\headheight=15pt

%\definecolor{mygray}{cmyk}{0,0,0,0.4}
\definecolor{mygray}{gray}{0.5}

\rhead{\textcolor{mygray}{\fontsize{10}{12}\selectfont DEEP LEARNING FOR COMPUTER VISION}}
\renewcommand{\headrule}{{\color{mygray}%
       \hrule width\headwidth height\headrulewidth \vskip-\headrulewidth}
}%renewcommand headrule

\renewcommand\textfraction{.02}
\renewcommand\floatpagefraction{.98}

% for rule under section heading
\newcommand {\sectionrule}{\vskip -0.9 cm
\color {mygray} \rule [0 cm] {17 cm}{0.1 mm} \color {black}}

%%%%%%%%%%%%% START %%%%%%%%%%%%% 
\date{}

\title{HW1 \\ Classification Task on Cifar-10}

\author{Bc. Martin Vozár}

\begin{document}
\maketitle
% \thispagestyle{fancy}

\section{Approach}
\sectionrule

The assignment proposes working with a standardized dataset on a 
straight-forward task and encourages trying multiples of different
configurations. First step in the process should be finding and choosing
appropriate configuration as a baseline.

For those purposes, we developed a handy framework for setup
of various configurations, similar to hyperparameter optimization
(see \emph{configs/*} for more details).
We are plotting $\text{torch.nn.CrossEntropyLoss}$ as our loss function,
and accuracy as our metric for both $\text{Training}$ and $\text{Test set}$.
Plots are made on a $\text{log}/\text{log}$
scale, as it allows easier interpretation and detections of potential
issues (see \emph{log2png.py} for more details).

After finding the baseline, we further explore chosen configuration
in effort to maximize accuracy.

The code was ran in \emph{conda} environment on \emph{Ubuntu 22.04} using
a single \emph{NVIDIA GTX 1070 Ti}. 

\section{Preliminary exploration}
\sectionrule

In this stage, we are mostly comparing different optimizers 
(SGD, SGD with Momentum=0.96, Adam, and AdamW) with
different learning rates. We are using defaultvalue for batch size
(128), and minimal data augmentation (RandomFlipVertical(p=0.5),
RandomFlipHorizontal(p=0.5)). We are using LeakyReLU as the default
activation function.

During optimization we perform gradient clipping and use
weight decay argument for all optimizers as means of regularization.

For each plot, we plot horizontal lines for maximum Test Accuracy
and minimum Test Loss in the respective plots.

\subsection{Naive - FFCN}

As a first choice meant to calibrate the general setup of
different optimizers, as well as gaining intuition for necessary
complexity of further examined Neural Networks, we tested a simple
FCNN (further as a nickname - Naive). For this Network we used
nn.Dropout(p=0.3).

Iteratively, we tuned individual learning rates for each optimizer
(as their behaviour varied quite a bit) as well as the Network
width and depth. 

Plotted are results for a Network:
\begin{itemize}
  \item nn.Linear(3072, 256)
  \item (nn.Linear(256, 256)) * 8
  \item nn.Linear(256, num\_classes)
\end{itemize}
with activation function after each but last layer. Torch
implementation of CrossEntropyLoss allows us to omit applying
Softmax on output layer, as well as one-hot encoding on the
labels.

\begin{figure}[H]
  \includegraphics[width=\textwidth, trim={0, 3.5cm, 0, 0}, clip]{../logsDense.png}
  \caption{Plot of results of the preliminary exploration of FCNN
  architecture.}
\end{figure}

\newpage

In those results, we clearly observe examples of overfitting in
most of the runs.

The second thing we can observe is a stagnation of SGD optimization.
Somewhat surprisingly, variants using Momentum stagnated for longer
period of time before proceeding to overfit.
However, this could be acredited to other factors, e.g.
specific initialization of weights.

\subsection{Convolutional Encoder}

We tested a few different variants of Convolutional Encoder. All variants
consisted of blocks of (nn.Conv2d layers, nn.MaxPool2d) and a final
nn.Linear layer at the end. Variables were strides of the convolutions
and number of blocks.

Plotted is the variant which achieved the target accuracy of 70\%
using SGD with Momentum as optimizer. We used this variant in the
following architectures as the final encoding component.

\begin{figure}[H]
  \includegraphics[width=\textwidth, trim={0, 3.5cm, 0, 0}, clip]{../logsConvE.png}
  \caption{Plot of results of the preliminary exploration of ConvEncoder
  architecture.}
\end{figure}

\newpage

\subsection{Residual Convolutional Encoder}

Drawing inspiration from ResNet architecture, we implemented a
basic adaptation of the principle.

The basic idea of the architecture:
\begin{itemize}
  \item nn.Conv2d(in\_channel=3, out\_channels=16, kernel\_size=1, stride=1)
  \item N x ResBlock
  \item ConvEncoder
\end{itemize}

The first layer expands the number of channels to a set number,
which then remains unchanged passing through the ResBlocks. Finally,
Convolutional Encoder (adapted to also handle in\_channels different
from original images) outputs the logits.

From multiple tried and tested variants of ResBlock, we settled on definition:
\begin{itemize}
  \item input x
  \item z = nn.Conv2d(in\_channels=16, out\_channels=32)(x)
  \item z = nn.BatchNorm2d(num\_features=32)(z)
  \item z = activation(z)
  \item z = nn.Conv2d(in\_channels=32, out\_channels=16)(z)
  \item z = nn.BatchNowm2d(num\_features=16)(z)
  \item output x = x + z
\end{itemize}

It has been argued (source: some internet forum) that using Dropout and 
BatchNorm at the same time
can lead to issues during the training. However, we have not encountered
(or at least identified) any artifacts, and BatchNorm can boost
convergene and regularization.

Various number of in/out\_channels were tested and those values were
chosed as a reasonable compormise between convergence speed
and potential for accuracy.
We used kernel\_size=3, stride=1, padding=1 bias=False in both 
convolutional layers.

For this configuration, we first tested for N=4 (\# of ResBlocks)
to find the best performing configuration. Then, we varied the
depth of the Network, as well as other parameters.

In Figure 3., we observe an increase to accuracy 77\%. The most accurate
Network seems to have already plateaued, signifying approaching limit
of this configuration.

Interpreting the behaviour of the optimizers, we select SGD with Momentum
and AdamW for further examination. We observe similar behvaiour as with
ConvEncoder. To explore further, we vary the num\_blocks in the RN
architecture, and vary learning rates and weight decay with selected
optimizers.

\begin{figure}[H]
  \includegraphics[width=\textwidth, trim={0, 3.5cm, 0, 0}, clip]{../logsRN-04.png}
  \caption{Plot of results of the preliminary exploration of RN04 with ConvEncoder
  architecture.}
\end{figure}

We achieved further improvement of Test accuraccy. However, with the
individual runs getting more and more similar encouraged us to change
plots to linear scale.

Results from comparison of weight\_decay values give a slight hint
of improved regularization.

\begin{figure}[H]
  \includegraphics[width=\textwidth, trim={0, 3.5cm, 0, 0}, clip]{../logsRN.png}
  \caption{Closer examination of RN architecture variants.}
\end{figure}

In the following run, we also examined Tanh and Sigmoid together with LeakyReLU with the same
set of configurations. We further examined different 
weight\_decay values with expectation of better regularization.
This was not observed. We prematurely stopped the run after 50 epochs,
as we identified overfitting even in worse performing configurations.

For a few exploratory runs, we decided to stick to RN16, as for reducing
number of efficient LR values. We also compared Sigmoid, with not much
success ().

\begin{figure}[H]
  \includegraphics[width=\textwidth, trim={0, 3.5cm, 0, 0}, clip]{../logsRN16_0.png}
  \caption{Even closer examination of RN architecture variants.}
\end{figure}

\begin{figure}[H]
  \includegraphics[width=\textwidth, trim={0, 3.5cm, 0, 0}, clip]{../logsRN16_1.png}
  \caption{Examination of Sigmoid activation performance against LeakyReLU baseline.}
\end{figure}


\newpage

\section{Regulization and data augmentation}
\sectionrule

It should be mentioned, that with Vision Transformers scoring 99.5\% accuracy
on Cifar-10 benchmark one could try using Transformers. As of now, the
CNNs do not seem to have reached their limist, yet, so we will stick to them.

One of the goals is to be able to stick to small-sized models, for speed and
general practicality. Most effort is improving the regularization of
the optimization.

We implemented a heavy\_regularization option
for data, applying more transformations on the inputs. weight\_regularization
does not seem to have the desired effect, which has not deterred s from exploring it further.
We also compared various varians of Linear Function, with relatively high
weight decay value. We observe less overfitting, but also a slight. Our
explanation is the high value of weight\_decay.

\begin{figure}[H]
  \includegraphics[width=\textwidth, trim={0, 3.5cm, 0, 0}, clip]{../logsRN16_2.png}
  \caption{Examination of variations of LU variants performance against LeakyReLU baseline.}
\end{figure}

\newpage

In the Figure 7. run, we also tested changing batch\_size. The value we chose was
16, just to examine the effect of lower values. The effect we ascribed to this
was perhaps a bit more unstable validation loss. 

It is difficult to pinpoint the exact cause of individual effects. Further,
we suspect a small batch\_size can cause issues with heavy data augmentation.
We will keep it high, as to take gradient from more samples with more
representation of various augmentations.
Another thing realized at this stage is mistakenly using torch.nn.Dropout
where torch.nn.Dropout2d was supposed to be used.

For the following run, we included nn.Dropout2d(p=1/3)
Further, based on a suspicion that heavy masking might be an issue
for convolutional layers, we started thinking about a more sophisticated
design with more skip connections.

The idea is, that all blocks receive the original input stacked
as extra channels.
Though we wanted to keep the architecture as similar to the tested
RN16 variant, some adaptations were made, as to accomodate the extra input
channels. The extra input channels were ignored on
the residual operation: \emph{(x = x[:, :16] + z)}.

To see, if we were going in the right direction, we tested both RN16 from
previous run against the new architecture (nicknamed RR16) on the
heavily augmented dataset.

\subsection{Pre-training with masks}

In the following runs, we iteratively selected random masking
with 4x4 granularitya and $p=0.5, \ p=0.7, \ p=0.9$ variants.
We implented them using using 
\emph{torchvision.transforms.v2.Lambda}. The masks randomly
zero out a portion of the image in all channels.

The idea was to partially reproduce approach of Masked AutoEncoders.
To reduce training time, as well as to stay focused on the
task, we did not implement the Decoder, and remained using
only the Encoder as previously.

We also continued optimizing the learning rate and weight decay
values. As the tested architectures are almost identical, we
continued using the values found to be efficient for RN also
for the RR adaptation. In one of the runs, we also compared
RR16 to RR32.

Though we see a decline in the accuracy, we consider this
a step in the right direction. It seems that at this stage
we might have over-regularized. More importatnly, we narrowed
down the range of efficient learning rate values.

\begin{figure}[H]
  \includegraphics[width=\textwidth, trim={0, 3.5cm, 0, 0}, clip]{../logsRR_2.png}
  \caption{Comparison of RN and RR variants, comparison of narrower
  range of learning rate and weight decay values.}
\end{figure}

\begin{figure}[H]
  \includegraphics[width=\textwidth, trim={0, 3.5cm, 0, 0}, clip]{../logsRR_3.png}
  \caption{Comparison of RR16 and RR32 variants in a narrower range
  of learning rate and weight decay values.}
\end{figure}

\subsection{Fine-Tuning}

In the following runs, we train the best performing RR32
variant (in terms of accuracy), together with a variant
with no weight decay.

In the first stage, we will train the model on heavily masked
inputs, using $p=0.9$, with various granularity, 
$4x4, \ 2x2, \ 1x1$ blocks. (Using the $1x1$ block is
comparable to nn.Dropout, with the difference of all the
channels having the same pixels blocked).

We regularly save the model state, and later fine-tune on
RandomCrop augmentation.

\end{document}

 
